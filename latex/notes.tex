\documentclass[a4paper]{article}
%% Language and font encodings
\usepackage[english]{babel}
\usepackage[utf8x]{inputenc}
\usepackage{booktabs}
\usepackage{tabu}
\usepackage[T1]{fontenc}
%% Sets page size and margins
\usepackage[a4paper,top=2cm,bottom=2cm,left=1cm,right=1cm,marginparwidth=1cm]{geometry}
%% Useful packages
\usepackage{amsmath}
\usepackage{graphicx}
%\usepackage{apacite}
\usepackage[colorinlistoftodos]{todonotes}
\usepackage[colorlinks=true, allcolors=blue]{hyperref}

\usepackage{physics}



\title{\vspace{-2cm} Table ronde weak lensing calibration}
\author{Olivier Kauffmann}

\begin{document}
\maketitle



\section{Abstract}

Observed ellipticity of one individual galaxy with properties $\vec{P}$ (Eq.10 in Pujol+2018)
$$
\vec{e}^{obs} = R(\vec{P})\vec{g} +\vec{a}(\vec{P}) + f(\vec{e}^I)
$$

Response (Eq.2 in Pujol+2018)
$$
R_{\alpha\beta} = \dfrac{\partial e_\alpha^{obs}}{\partial g_\beta} \approx \dfrac{e_\alpha^{obs,+}-e_\alpha^{obs,-}}{2\Delta g_\beta}
$$


Taylor expansion of the exact expression from theory (for $|g|\leq1$)
$$
e = \dfrac{e^I+g}{1+g^*e^I} = e^I+g-g^*(e^I)^2 +O(g^2)
,\quad \text{if } |g^*e^I|<1 \text{ as } \dfrac{1}{1+x}=\sum_{k=0}^{\infty}(-1)^kx^k \text{ for } |x|<1 
$$

$$
\vec{e} = \vec{e}^I+\vec{g}(1+|\vec{e}^I|^2)-2(\vec{e}^I\cdot \vec{g})\vec{e}^I +O(g^2)
$$

$$
\vec{e} = \vec{e}^I +
\left(
\begin{array}{cc}
   1-(e_1^I)^2+(e_2^I)^2 & -2e_1^I e_2^I \\
   -2e_1^I e_2^I & 1+(e_1^I)^2-(e_2^I)^2 
\end{array}
\right)
\vec{g}= \vec{e}^I + A(\vec{e}^I)\vec{g}
$$


Observed ellipticity of one individual galaxy with the next order term
$$
\vec{e}^{obs} = R(\vec{P})A(\vec{e}^I)\vec{g} +\vec{a}(\vec{P}) + f(\vec{e}^I)
$$

$$
\dfrac{\partial e_\alpha^{obs}}{\partial g_\beta} = \left[R(\vec{P})A(\vec{e}^I)\right]_{\alpha\beta} = \widetilde{R}(\vec{P},\vec{e}^I)_{\alpha\beta}
$$


\section{Update 2019}
\subsection{Predictions}

Taylor expansion around $g=0$
$$
e_\alpha^\text{true} = e_\alpha^I + A_{\alpha\beta}g_\beta + B_{\alpha\beta\gamma}g_\beta g_\gamma + O(g^3)
$$
The mean property $\langle e^\text{true}\rangle=g$ holds whatever the degree of the expansion (see proof below).

$$
\begin{array}{rcl}
A_{11} &=&  1-(e_1^I)^2+(e_2^I)^2 \\
A_{12}=A_{21} &=&  -2e_1^I e_2^I \\
A_{22} &=&  1-(e_1^I)^2+(e_2^I)^2 \\
\end{array}
$$

$$
\begin{array}{rcl}
B_{111} &=&   (e_1^I)^3 -3e_1^I(e_2^I)^2 -e_1^I \\
B_{112}=B_{121} &=& 3(e_1^I)^2e_2^I -(e_2^I)^3 \\
B_{122} &=&  -(e_1^I)^3 +3e_1^I(e_2^I)^2 -e_1^I \\
B_{211} &=&  -(e_2^I)^3 +3(e_1^I)^2e_2^I -e_2^I \\
B_{212}=B_{221} &=& 3e_1^I(e_2^I)^2 -(e_1^I)^3 \\
B_{222} &=&   (e_2^I)^3 -3(e_1^I)^2e_2^I -e_2^I \\
\end{array}
$$

\subsection{Observations}

Observed ellipticity defined as the true ellipticity with a multiplicative response and an additive bias, both functions of the galaxy physical properties $\vec{P}$
$$
\begin{array}{lcl}
e_\alpha^\text{obs} &=& R_{\alpha\beta}e_\beta^\text{true} + a_\alpha + S \\
&=& R_{\alpha\beta}e_\beta^I + R_{\alpha\beta}A_{\beta\gamma}g_\gamma + R_{\alpha\beta}B_{\beta\gamma\delta}g_\gamma g_\delta + a_\alpha + S \\
&=& \tilde{R}_{\alpha\beta}g_\beta + \dfrac{1}{2}\tilde{Q}_{\alpha\beta\gamma}g_\beta g_\gamma + a_\alpha + f(e^I)
\end{array}
$$
with $\tilde{R}_{\alpha\beta} = R_{\alpha i}A_{i \beta}$ (4 independent values) and $\tilde{Q}_{\alpha\beta\gamma} = 2R_{\alpha i}B_{i \beta\gamma}$ (6 independent values)

First derivatives (4 equations):
$$
\dfrac{\partial e_\alpha^\text{obs}}{\partial g_\beta} = \tilde{R}_{\alpha\beta} + \dfrac{1}{2}\left( \tilde{Q}_{\alpha\beta\gamma}g_\gamma + \tilde{Q}_{\alpha\beta\gamma}g_\beta \right)
$$

Second derivatives (6 equations):
$$
\dfrac{\partial^2 e_\alpha^\text{obs}}{\partial g_\beta\partial g_\gamma} = \dfrac{1}{2}\left( \tilde{Q}_{\alpha\beta\gamma} + \tilde{Q}_{\alpha\gamma\beta} \right)
$$

Finite differences for first derivatives, with 2 points in the $(g_1,g_2)$ plane:
$$
\dfrac{\partial e_\alpha^\text{obs}}{\partial g_\beta} \approx \dfrac{e_\alpha^\text{obs}(+\Delta g_\beta)-e_\alpha^\text{obs}(-\Delta g_\beta)}{2\Delta g_\beta}
$$

Finite differences for second derivatives, with 5 points in the $(g_1,g_2)$ plane:
$$
\dfrac{\partial^2 e_\alpha^\text{obs}}{\partial g_\beta\partial g_\gamma} \approx \dfrac{e_\alpha^\text{obs}(+\Delta g_\beta)+e_\alpha^\text{obs}(-\Delta g_\beta)+e_\alpha^\text{obs}(+\Delta g_\gamma)+e_\alpha^\text{obs}(-\Delta g_\gamma) - 4e_\alpha^\text{obs}(0)}{4\Delta g_\beta\Delta g_\gamma}
$$


\section{Proof of the mean property}

This is the proof of that the mean of the ellipticity estimator $e^\text{true}$ is $g$. The mean is taken over intrinsic ellipticities $e^I$ with probability density $p(e^I)$ assumed to be isotropic.

Proof with exact expression

$$
\begin{array}{rcll}
\langle e^\text{true}\rangle
&=& {\displaystyle \int_{D(\vec{0},1)} e^\text{true}(e^I,g) p(e^I) de^I} & \text{$D(\vec{0},1)$ is the disk of center $\vec{0}$ and radius $1$ (hereafter $C(0,1)$ the circle)} \\
&=& {\displaystyle \int_0^1 \left( \int_0^{2\pi} e^\text{true}(ye^{2i\phi},g) d\phi \right) p(y)ydy } 
& \text{$e^I=ye^{2i\phi}$, $p(e^I)$ must be polar symmetric} \\
&=& {\displaystyle \int_0^1 \left( \int_0^{2\pi} \dfrac{ye^{2i\phi}+g}{1+g^*ye^{2i\phi}} d\phi \right) p(y)ydy } & \\
&=& {\displaystyle \int_0^1 \left( -i \oint_{C(0,1)} \dfrac{yu+g}{u(1+g^*yu)} du \right) p(y)ydy } & \text{$u=e^{2i\phi}$, $\dfrac{du}{d\phi}=2iu$, $\dfrac{d\phi}{du}=\dfrac{-i}{2u}$ but as $\phi\in[0,2\pi)$, there are two circles } \\
&=& 2\pi g {\displaystyle \int_0^1 p(y)ydy } & \text{by residue theorem, with poles $u=0$ (inside) and $u=-1/(g^*y)$ (outside)} \\
&=& g & \text{imposed by normalization of $p$} \\
\end{array}
$$

Proof with Taylor expansion

$$
\begin{array}{rcll}
\langle e^\text{true}\rangle
&=& {\displaystyle \int_0^1 \left( -i \oint_{C(0,1)} \dfrac{yu+g}{u(1+g^*yu)} du \right) p(y)ydy } & \\
&=& {\displaystyle \int_0^1 \left( -i \oint_{C(0,1)} y+\dfrac{g}{u}-g^*y^2u+\sum_{k=2}^\infty f_k u^k du \right) p(y)ydy } & \text{with $f_k$ some factors} \\
&=& 2\pi g {\displaystyle \int_0^1 p(y)ydy } & \text{as } {\displaystyle \oint_{C(0,1)} z^m dz =\left\{\begin{array}{ll} 2\pi i & m=-1 \\ 0 & \text{otherwise} \end{array}\right.} \\
&=& g & \\
\end{array}
$$


\end{document}